\documentclass[a4paper,12pt]{article}

\usepackage[dutch]{babel}
\usepackage[T1]{fontenc}
\usepackage{fontspec}

\setmainfont{Nimbus Sans}

\title{Voorbeeld {\LaTeX}}
\author{Knuth, D.}
\date{\today}

\begin{document}

\maketitle

Ik ben makelaar in koffie, en woon op de Lauriergracht No 37. Het is mijn gewoonte niet, romans te schrijven, of zulke dingen, en het heeft dan ook lang geduurd, voor ik er toe overging een paar riem papier extra te bestellen, en het werk aan te vangen, dat gij, lieve lezer, zoëven in de hand hebt genomen, en dat ge lezen moet als ge makelaar in koffie zijt, of als ge wat anders zijt. Niet alleen dat ik nooit iets schreef wat naar een roman geleek, maar ik houd er zelfs niet van, iets dergelijks te lezen, omdat ik een man van zaken ben.

Sedert jaren vraag ik mij af, waartoe zulke dingen dienen, en ik sta verbaasd over de onbeschaamdheid, waarmee een dichter of romanverteller u iets op de mouw durft spelden, dat nooit gebeurd is, en meestal niet gebeuren kan. Als ik in mijn vak---ik ben makelaar in koffie, en woon op de Lauriergracht No 37---aan een principaal---een principaal is iemand die koffie verkoopt---een opgave deed, waarin maar een klein gedeelte der onwaarheden voorkwam, die in gedichten en romans de hoofdzaak uitmaken, zou hij terstond Busselinck \& Waterman nemen.

Dat zijn ook makelaars in koffie, doch hun adres behoeft ge niet te weten. Ik pas er dus wel op, dat ik geen romans schrijf, of andere valse opgaven doe. Ik heb dan ook altijd opgemerkt dat mensen die zich met zoiets inlaten, gewoonlijk slecht wegkomen. Ik ben drieënveertig jaar oud, bezoek sedert twintig jaren de beurs, en kan dus voor de dag treden, als men iemand roept die ondervinding heeft. Ik heb al wat huizen zien vallen! En gewoonlijk, wanneer ik de oorzaken naging, kwam het me voor, dat die moesten gezocht worden in de verkeerde richting die aan de meesten gegeven was in hun jeugd.

\end{document}